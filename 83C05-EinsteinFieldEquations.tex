\documentclass[12pt]{article}
\usepackage{pmmeta}
\pmcanonicalname{EinsteinFieldEquations}
\pmcreated{2013-03-22 15:02:34}
\pmmodified{2013-03-22 15:02:34}
\pmowner{rspuzio}{6075}
\pmmodifier{rspuzio}{6075}
\pmtitle{Einstein field equations}
\pmrecord{25}{36757}
\pmprivacy{1}
\pmauthor{rspuzio}{6075}
\pmtype{Topic}
\pmcomment{trigger rebuild}
\pmclassification{msc}{83C05}
\pmsynonym{Einstein's GR equations}{EinsteinFieldEquations}
%\pmkeywords{Einstein's GR equations}
%\pmkeywords{quantum spece-times}
%\pmkeywords{quantum Riemannian geometry}
\pmrelated{CategoryOfRiemannianManifolds}
\pmrelated{PseudoRiemannianManifold}
\pmrelated{QuantumGeometry2}
\pmrelated{QuantumSpaceTimes}
\pmrelated{QuantumGravityTheories}
\pmrelated{RiemannianMetric}
\pmrelated{MathematicalProgrammesForDevelopingQuantumGravityTheories}
\pmrelated{MathematicalProgrammesForDevelopingQuantumGravityTheories}

\endmetadata

% this is the default PlanetMath preamble.  as your knowledge
% of TeX increases, you will probably want to edit this, but
% it should be fine as is for beginners.

% almost certainly you want these
\usepackage{amssymb}
\usepackage{amsmath}
\usepackage{amsfonts}

% used for TeXing text within eps files
%\usepackage{psfrag}
% need this for including graphics (\includegraphics)
%\usepackage{graphicx}
% for neatly defining theorems and propositions
%\usepackage{amsthm}
% making logically defined graphics
%%%\usepackage{xypic}

% there are many more packages, add them here as you need them

% define commands here
\begin{document}
\section{Introduction and Definition}

The \emph{Einstein Field Equations} are the fundamental equations of Einstein's
general theory of relativity.  For a description of this physical theory and of
the physical significance of solutions of these \PMlinkexternal{equations}{http://planetphysics.org/encyclopedia/TopicOnEquationsInMathematicalPhysics.html}, please see 
\PMlinkexternal{PlanetPhysics}{http://planetphysics.org/}.  Here, we shall discuss the mathematical properties of these
equations and their relevance to various branches of pure mathematics.

The Einstein field equations are a system of second order coupled nonlinear partial differential equations for a Riemannian metric tensor on a Riemannian manifold.  Let $M$ be
a differentiable manifold and let $T_{\mu \nu}$ and $g_{\mu \nu}$ be symmetric
tensor fields \footnote{Throughout this entry, we shall use index notation
for tensor fields because that is common in the literature (especially physics literature) and is convenient for computation of particular solutions.  
Moreover, we shall, fittingly enough, employ Einstein's summation convention.}.  Further, assume that $g_{\mu \nu}$ is invertible on a dense subset of $M$ and twice differentiable.  (It is possible to relax the latter requirement by interpreting the equations distributionally.)  Then the Einstein equations 
read as follows:
\footnote{In the physics literature, the coefficient of $T_{\mu \nu}$ is
written as $\frac{8\pi G}{c^4}$, where $G$ is the gravitational constant, 
$c$ is the light velocity constant but, since we are interested in the purely
mathematical properties of these equations, we shall set $G = c = 1$ here,
which may be accomplished by working in a suitable set of physical units.
It might also be worth mentioning that, in physics, the tensor $T_{\mu \nu}$ 
is the stress-energy tensor, which encodes data pertaining to the mass, energy, and momentum densities of the surrounding space.  The number $\Lambda$
is known as the cosmological constant because it determines large-scale
properties of the universe, such as whether it collapses, remains stationary, or expands.}
 \[G_{\mu \nu} = \Lambda g_{\mu \nu} + 8 \pi T_{\mu \nu}\]
Here, $G_{\mu\upsilon}=R_{\mu\upsilon}-\frac{1}{2}g_{\mu\upsilon}R$
is the \emph{Einstein Tensor}, $R_{\mu\upsilon}$ is the Ricci tensor, and 
$R=g^{\mu\nu}R_{\mu\nu}$ is the Ricci scalar, and $g^{\mu\nu}$ is the 
inverse metric tensor.  

One possibility is that the tensor field $T_{\mu \nu}$ is specified and that
these equations are then solved to obtain $g_{\mu \nu}$.  A noteworthy case of 
this is the \emph{vacuum Einstein equations}, in which $T_{\mu \nu} = 0$.
Another possibility is that $T_{\mu \nu}$ is given in terms of some other
fields on the manifold and that the Einstein equations are augmented by
differential equations which describe those fields.  In that case, one speaks
of Einstein-Maxwell equations, Einstein-Yang-Mills equations, and the like
depending on what these other fields may happen to be.  It should be noted 
that, on account of the Bianchi identity, there is an integrability condition
$\nabla_\mu (g) T^{\mu \nu} = 0$.  (Here, $\nabla (g)$ denotes covariant
differentiation with respect to the Levi-Civita connection of the metric
tensor $g_{\mu \nu}$.)  When choosing $T_{\mu \nu}$, these conditions must
be taken into account in order to guarantee that a solution is possible.

\section{Diffeomorphism Invariance}

Because they are constructed from tensors, the Einstein equations have an
important invariance property.  Suppose that $g_{\mu \nu}$ and $T_{\mu \nu}$ satisfy the Einstein equations.  Then, for any diffeomorphism $f \colon M \to
M$, we also have that $(f^* g)_{\mu \nu}$ and $(f^* T)_{\mu \nu}$ also 
satisfy the Einstein equations.  (Here, the notation $f^*$ denotes pullback 
with respect to the diffeomorphism $f$.)

This fact means that we must be careful when talking about specifying solutions
by boundary conditions.  Usually, when dealing with a differential equation,
we would expect that we could specify a solution uniquely by providing enough
boundary data.  Here, however, this will not work since we could find a
diffeomorphism which reduces to the identity near the boundary but differs
from the identity elsewhere and use that to produce another solution which 
would satisfy the same boundary conditions.  What one should do instead is to
consier equivalence classes of solutions modulo diffeomorphism and only
ask that boundary conditions specify solutions up to diffeomorphisms.  As we
shall see later, with such an understanding, one can indeed specify solutions 
in terms of initial data.

In order to adress this issue and to be able to treat the Einstein equations
much as one would treat other differential equations, a common practise is to
supplement the Einstein equations with auxiliary condidtions which serve to
define a coordinate system and hence single out a particular element of an
equivalence class in diffeomorphism.  While such auxiliary equations should
ideally single out a representative for each equivalence class, in practise,
one is content with considerably less --- a particular choice auxiliary 
conditions might only work with some solutions or may only specify a 
subset of an equivalence class with more than one element. 

\textbf{Remarks:}
The major obstruction to the GR theory is that Einstein's GR equations--although solvable in principle--
are readily solvable only in special cases, with specified boundary conditions. The bigger
problem is the difficulty of formulating quantum field theories (QFT) in a manner
which is logically consistent with Einstein's GR formulation so that a valid Quantum Gravity (QG) theory
is formulated that yields results consistent with both GR and quantum theories in the presence of intense
gravitational fields. So far, encouraging results have been obtained only for the limiting case
of low intensity gravitational fields as in S. Weinberg's algebraic approach to QFT  and QG using supersymmetry and 
graded `Lie' algebras or superalgebras. 

\section{Hyperbolic Formulations}

\section{Variational Principles}

\section{Alternative Formulations}

 An 
\PMlinkexternal{alternative, more general formulation of GR and GR Field Equations}{http://planetphysics.org/?op=getobj&from=objects&id=441} would involve a categorical framework, such as the category of pseudo-Riemannian manifolds, and/or the category of Riemannian manifolds, with, or without, a Riemannian metric. Expanding universes and black hole singularities,
with or without hair, either with an event horizon, or `naked', can be treated within
such an unified categorical framework of Riemannian/ pseudo-Riemanian manifolds and their 
transformations represented either as morphisms or by functors and natural transformations
between functors. Quantized versions in quantum gravity may also be available based on
spin foams represented by time-dependent/ parameterized functors between spin networks
including extremely intense, but finite, gravitational fields.  
\section{Global Structure}

\section{Initial Value Formulation}

\section{Special Solutions}

\subsection{Spatially Homogeneous Solutions}

\subsection{Solutions with Symmetries}

\subsection{Algebraically Special Solutions}

\subsection{Linearization}

\subsection{Singularities}

\subsection{Asymptotically Flat Solutions}

\subsection{Existence Theorems}

%%%%%
%%%%%
\end{document}
